%%%%%%%%%%%%%%%%%%%%%%%%%%%%%%
%%%%%%%%%%%% Preamble %%%%%%%%%%%%%
%%%%%%%%%%%%%%%%%%%%%%%%%%%%%%

% Word target: 2,000 words

% Outline:
% I. Introduction (250 words)
% II. Generalized Maximum Entropy Estimator (500 words) (~475 currently)
% III. Monte Carlo Experiment (750 words)
% IV. Application (250 words)
% V. Conclusions (250 words)

% Declare document class and miscellaneous packages
\documentclass[english]{article}
\usepackage{natbib}
\usepackage{amsmath}
\usepackage{mathrsfs}
\usepackage{amssymb}
\usepackage{ctable}
\usepackage{setspace}
\usepackage{longtable}
\usepackage{url}
\usepackage{moredefs,lips} 
\usepackage{IEEEtrantools}
\usepackage{multirow}
\usepackage{enumerate}
\usepackage[normalsize]{caption}
\usepackage{afterpage}
\usepackage[all]{nowidow}
\usepackage{listings}
%\usepackage{fullpage}
\urlstyle{rm}

%Hyper-references
\usepackage{hyperref}
\hypersetup{colorlinks, citecolor=black, filecolor=black, linkcolor=black, 
urlcolor=black, pdftex}

% Title page
\title{Improving Small Sample Performance of the Generalized Maximum 
Entropy Estimator}
\author{
Heath Henderson\thanks{Corresponding author; 1300 New York Avenue 
NW, Washington, DC 20577; Tel: +1 202 623 3860; Fax: +1 202 312 4202; 
Email: heathh@iadb.org.}\\
\textit{Office of Strategic Planning and Development Effectiveness} \\
\textit{Inter-American Development Bank} \\
\\
Amos Golan \\
\textit{Department of Economics}\\
\textit{American University} \\
\\
Skipper Seabold \\
\textit{Department of Economics}\\
\textit{American University}
\\ \\
}

\date{\today}

\begin{document}

%%%%%%%%%%%%%%%%%%%%%%%%%%%%%%
%%%%%%%%% Title Page and Abstract %%%%%%%%%
%%%%%%%%%%%%%%%%%%%%%%%%%%%%%%

\begin{titlepage}
\maketitle

\begin{abstract}
%\textit{Key words}: \\
%\textit{JEL codes}:  
\end{abstract}
\thispagestyle{empty}
\end{titlepage}
\newpage

% Set citations without comma between author and year
\bibpunct{(}{)}{;}{a}{}{,}

\doublespacing

% Outline:
% I. Introduction (250 words)
% II. Generalized Maximum Entropy Estimator (250 words)
% III. Monte Carlo Experiment (750 words)
% IV. Application (500 words)
% V. Conclusions (250 words)

%%%%%%%%%%%%%%%%%%%%%%%%%%%%%%
%%%%%%%%%%% Introduction %%%%%%%%%%%%
%%%%%%%%%%%%%%%%%%%%%%%%%%%%%%

\section*{Introduction}
\label{sec: intro}

%%%%%%%%%%%%%%%%%%%%%%%%%%%%%%
%%%%%%%%%% GME Estimator  %%%%%%%%%%%
%%%%%%%%%%%%%%%%%%%%%%%%%%%%%%

\section*{GCE Estimator}
\label{sec: gme}

Following \cite{golan1996} and \citet{golan2008}, consider the following 
linear regression model: 
\begin{equation}
\mathbf{y} = \mathbf{X\beta} + \mathbf{\varepsilon}
\end{equation}

\noindent
where $\mathbf{y}$ is a $T$-dimensional vector of observations on the 
dependent variable, $\mathbf{X}$ is a $T\times K$ matrix of exogenous
variables, $\mathbf{\beta}$ is a $K$-dimensional vector of unknown 
parameters, and $\mathbf{\varepsilon}$ is a $T$-dimensional vector of 
random errors.
Each $\mathbf{\beta}_k$ and $\mathbf{\varepsilon}_t$ in the GCE 
framework is typically viewed as the mean value of some well-defined 
random variable, which we denote as $\mathbf{z}_k$ and $\mathbf{v}_t$, 
respectively.
Accordingly, let $\mathbf{p}_k$ be an $M$-dimensional proper 
probability distribution defined on the support $\mathbf{z}_k$ such that 
$\mathbf{\beta}_k = \sum_m p_{km}z_{km} = \mathbf{z}_k' 
\mathbf{p}_k$.
Similarly, let $\mathbf{w}_t$ be a $J$-dimensional proper probability 
distribution defined on the support $\mathbf{v}_t$ such that  
$\mathbf{\varepsilon}_t = \sum_j w_{tj}z_{tj} = \mathbf{v}_t' 
\mathbf{w}_t$.

Without loss of generality, the linear regression model can then be 
reparameterized as follows:
\begin{equation}
\mathbf{y} = \mathbf{X\beta} + \mathbf{\varepsilon} = 
\mathbf{X Z p} + \mathbf{V w}
\end{equation}

\noindent
where 
\begin{equation}
\mathbf{\beta} = \mathbf{Zp} = 
\left[ \begin{array}{cccc}
\mathbf{z}_1' & 0 & \cdot & 0 \\
0  & \mathbf{z}_2' & \cdot & 0 \\
\cdot  & \cdot  & \cdot & \cdot  \\
0 & 0 & \cdot & \mathbf{z}_K'
\end{array}\right]
\left[ \begin{array}{c}
\mathbf{p}_1 \\
\mathbf{p}_2 \\
\cdot \\
\mathbf{p}_K 
\end{array}\right]
\end{equation}

\noindent
and 
\begin{equation}
\mathbf{\varepsilon} = \mathbf{Vw} = 
\left[ \begin{array}{cccc}
\mathbf{v}_1' & 0 & \cdot & 0 \\
0  & \mathbf{v}_2' & \cdot & 0 \\
\cdot  & \cdot  & \cdot & \cdot  \\
0 & 0 & \cdot & \mathbf{v}_T'
\end{array}\right]
\left[ \begin{array}{c}
\mathbf{w}_1 \\
\mathbf{w}_2 \\
\cdot \\
\mathbf{w}_T 
\end{array}\right].
\end{equation}

\noindent
The dimensions of $\mathbf{Z}$ and $\mathbf{p}$ are then 
$K \times KM$ and $KM \times 1$, respectively, and the dimensions of 
$\mathbf{V}$ and $\mathbf{w}$ are $T \times TJ$ and $TJ \times 1$, 
respectively.
It should be noted that while it is possible to construct unbounded and 
continuous supports,%
\footnote{For example, see \citet{golan2002}.}
for the sake of simplicity the above support spaces are constructed as 
discrete and bounded. 

Let $\mathbf{q}$ be a $KM$-dimensional vector of prior weights for the 
parameters $\mathbf{\beta}$ with prior mean $\mathbf{Zq}$.
Analogously, let $\mathbf{u}$ be a $TJ$-dimensional vector of prior weights 
for the disturbances $\mathbf{\varepsilon}$ with prior mean $\mathbf{Vu}$.
The GCE estimator then selects $\mathbf{p}$, $\mathbf{w}$ $\gg$ 
$\mathbf{0}$ to minimize 
\begin{equation}
I({\mathbf{p}, \mathbf{q}, \mathbf{w}, \mathbf{u}}) = 
\mathbf{p}' \ln (\mathbf{p}/\mathbf{q}) + 
\mathbf{w}' \ln (\mathbf{w}/\mathbf{u})
\end{equation}

\noindent
subject to
\begin{equation}
\mathbf{y} = \mathbf{X Z p} 
+ \mathbf{V w}
\end{equation}
\begin{equation}
\mathbf{1}_K = (\mathbf{I}_K \otimes \mathbf{1}_M')\mathbf{p}
\end{equation}
\begin{equation}
\mathbf{1}_T = (\mathbf{I}_T \otimes \mathbf{1}_J')\mathbf{w}
\end{equation}

\noindent
where $\mathbf{1}$ denotes a vector of ones, $\mathbf{I}$ is the 
identity matrix, and $\otimes$ is the Kronecker product.
The reader is referred to \citet[Chap.\ 6]{golan1996} or 
\citet[Chap.\ 6]{golan2008} for analytical solutions, discussion of 
efficient techniques for computation of the GCE solutions via the 
unconstrained dual version of the problem, and issues of inference. 

Intuitively, the values of $\mathbf{p}$ and $\mathbf{w}$ that minimize
$I(\cdot)$ are those that, out of all probabilities satisfying the constraints, 
are ``closest'' to the researcher's chosen priors.
That is, $I(\cdot)$ is a measure of the discrepancy between two probability 
distributions and, for given prior probabilities, the objective is to find the 
posterior probabilities that minimize this discrepancy function.%
\footnote{See \citet[Chap.\ 6]{judge2011} for a full discussion of the 
cross entropy (or Kullback-Leibler) objective function.}
It can further be shown that GCE is a generalization of GME formalism.
That is, with uniform priors the GCE solution is identical to that which 
maximizes $H(\mathbf{p},\mathbf{w}) = - \mathbf{p}' \ln 
(\mathbf{p}) - \mathbf{w}' \ln (\mathbf{w})$ subject to the data
consistency and proper probability constraints. 

A central question in GCE estimation pertains to the choice of 
priors or reference distributions.
Prior information is frequently incomplete, noisy, or simply missing, and 
the researcher must choose among a possibly infinite number of 
reference distributions. 
While in such situations priors are often specified as uniform (i.e.\ 
$\mathbf{q}=1/M$ and $\mathbf{u}=1/J$), more informative 
reference distributions can enhance finite sample performance, especially 
for ill-posed, ill-conditioned, or noisy problems.%
\footnote{For example, see \citet{heckelei2003}.}
Accordingly, in the context of limited \textit{a priori} information, 
we seek a basis for selecting among alternative plausible reference 
distributions in an effort to improve the finite sample performance of 
the GCE estimator. 
To this end, we conduct Monte Carlo experiments to examine competing
measures or criteria by which informative priors might be chosen.

%%%%%%%%%%%%%%%%%%%%%%%%%%%%%%
%%%%%%%% Monte Carlo Experiment %%%%%%%%%
%%%%%%%%%%%%%%%%%%%%%%%%%%%%%%

\section*{Monte Carlo Experiment}
\label{sec: mce}

%%%%%%%%%%%%%%%%%%%%%%%%%%%%%%
%%%%%%%%%%% Application %%%%%%%%%%%%%
%%%%%%%%%%%%%%%%%%%%%%%%%%%%%%

\section*{Application}
\label{sec: app}

%%%%%%%%%%%%%%%%%%%%%%%%%%%%%%
%%%%%%%%%%% Conclusions %%%%%%%%%%%%%
%%%%%%%%%%%%%%%%%%%%%%%%%%%%%%

\section*{Conclusions}
\label{sec: conc}

%%%%%%%%%%%%%%%%%%%%%%%%%%%%%%
%%%%%%%%%%%% References %%%%%%%%%%%%
%%%%%%%%%%%%%%%%%%%%%%%%%%%%%%

%Start fresh page
\newpage
\cleardoublepage
\singlespacing

%Declare the style of the bibliography
\bibliographystyle{au-cms}

%Specify the file to use
\bibliography{/Users/hendersonhl/Documents/References}
\newpage

%%%%%%%%%%%%%%%%%%%%%%%%%%%%%%
%%%%%%%%%%%% Appendix %%%%%%%%%%%%%
%%%%%%%%%%%%%%%%%%%%%%%%%%%%%%

\newpage
\doublespacing
\section*{Appendix}


\end{document}